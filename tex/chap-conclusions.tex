%!TEX root = main.tex
\chapter{Concluding Thoughts}\label{ch:conclusions}

\section{Where have we come?}

We have tried in this book to present a particular picture of the world: everything is probability.  We started with basic definitions and applications, and followed the consequences of the rules of probability to examine more complex problems.  It is our hope that the reader sees that all of the analysis stems from a \emph{single} perspective.  In this way, one can approach \emph{any} problem of inference in a unified way, applying the  recipe we've used throughout:
\be
\i Propose a model for the data you observe (which could be as simple as ``there is an unknown true value for the observations'')
\i Specify your prior knowledge of the parameters in the model, in the form of a prior probability (which is often as simple as ``I don't know anything about the parameters, so all possible values are equally likely'')
\i Specify how likely your data would be if your model were true, which is the likelihood part of Bayes' rule
\i Apply the rules of probability, namely Bayes' rule, to determine the posterior probability for the parameters in the model
\i Use the properties of probability functions to calculate answers to specific questions, for example ``is it likely that this number is greater than zero?'' or ``are these two measurements different?''
\ee

Although I haven't covered all possible examples, and there are additions and clarifications still planned, this approach can be used for all new problems one faces.  The only steps that can be daunting, at times, is the mathematical consequences and even there we have seen that the judicious use of approximations can go a long way.

\section{Where are we going?}

Topics I'd love to add, and will when I have the chance, include (in no particular order),

\bi
\i Measurement in Science
\i Linear Regression and Correlation
\i Two-sample inferences
\i Classification
\i Model Building in Science
\i Analysis of Social Science Data
\i Inference for Deviation Parameters
\i Experimental Design
\i Computer simulations (e.g. MCMC)
\ei
